\documentclass[12pt,a4paper]{article}
\usepackage[margin=0.5in]{geometry}  
\title{數學思通}                          
\author{Ling-Hao Lin}                        
\date{}                               
\usepackage{xeCJK}                   
\setCJKmainfont{標楷體}               
\usepackage{amsthm,amssymb,amsmath}  
\usepackage{enumitem}
\usepackage{derivative}                 

\newtheorem*{theorem}{Theorem}
\newtheorem*{problem}{Problem}     
\newtheorem*{mydef}{Definition}    




\xeCJKsetup{CJKmath}                  %使數學式裡可以打中文字

\begin{document}
\maketitle

\begin{center}
\section*{SIR model}
\end{center}
\begin{align*}
\odv{S}{t}&= -\beta SI \\
\quad\\
\odv{I}{t}&= \beta SI-\gamma\\
\quad\\
\odv{R}{t}&= \gamma I\\
\quad
\end{align*}

\begin{itemize}
\item[S]: the susceptibles who are capable of catching the disease and becoming infected.
\item[I]: the infectives who have the disease and can transmit it.
\item[R]: the removed class consisting of the individuals who are recovered with immunity or dead due to the disease.
\item[$\beta$]: the infection transmission rate.
\item[$\gamma$]: the rate of recovery.
\item[N]: total population, $N=S+I+R$.
\item[t]: per day.
\end{itemize}


\quad
\begin{mydef}[$R_{0}$, basic reproduction number]
\qquad
$$R_{0}=\frac{\beta N}{\gamma}.$$
\qquad

\end{mydef}

\newpage

\subsection{求鑽石公主號和 2003 香港 SARS的 $R_{0}$,引用論文中的$\gamma$。}
\qquad	
 \begin{itemize}
 \item[方法 1.]
利用差分方程的方法估計$\beta$。
\qquad
    \begin{align*}
\odv{S}{t}\cong \frac{\Delta S}{\Delta t}&=\frac{S(t+1)-S(t)}{(t+1)-t}\\
\quad\\
                                         &=S(t+1)-S(t)=-\beta S(t)I(t)\\
\quad\\
                        &\Rightarrow\beta=\frac{S(t)-S(t+1)}{S(t)I(t)}.\\
     \end{align*}
   
 \item[方法 2.]
利用 Logistic function curve fitting 估計$\beta$。
\qquad
\begin{itemize}
  \item[$P(t)$]: the size of a population at time t.
  \item[K]: carrying capacity.
  \item[r]: intrinsic growth rate.
  \item[C]: constant.
\end{itemize}
\begin{align*}
\odv{P}{t}&=P(t)(r-\frac{r}{K}P(t))\\
\quad\\
\Rightarrow P(t)&=\frac{K}{1+Ce^{-rt}}.\\
\end{align*}

\end{itemize}



\subsection{觀察$R_{0}$}
\begin{itemize}
\item[T]: 病毒平均傳播天數。
\item[$I_{c}(m)$]: m天的累積病例數。
\end{itemize}

$$I_{c}(m)=1+R_{0}+R_{0}^2+...+R_{0}^n=\frac{R_{0}^{n+1}-1}{R_{0}-1}\mbox{ , }n=\frac{m}{T}.$$





\newpage

$$S\xrightarrow{{\beta SI}}I\xrightarrow{{\gamma I}}R$$

\end{document}